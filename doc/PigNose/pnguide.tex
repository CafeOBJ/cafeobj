%\documentclass[a4paper]{article}
\documentclass[a4paper,oneside]{memoir}
\usepackage[hscale=0.76,vscale=0.76]{geometry}
\setlength{\parindent}{0.0cm}
\setlength{\parskip}{1.4ex}
\usepackage{graphicx}
\usepackage{proof}
\usepackage{fancyvrb}
%%% Japanese
\usepackage{fontspec,xltxtra,xunicode}
\usepackage{indentfirst}

% freely available M+ fonts
\setmainfont[Mapping=tex-text]{M+ 2p}
\setsansfont[Mapping=tex-text]{M+ 2c}
\setmonofont[Mapping=tex-text]{M+ 2m medium}
%\setmainfont[Mapping=tex-text]{メイリオ}
%\setsansfont[Mapping=tex-text]{Hiragino Kaku Gothic ProN W3}
%\setmonofont[Mapping=tex-text]{Osaka-Mono}
%\setjamonofont{Osaka-Mono}
\XeTeXlinebreaklocale "ja"
\XeTeXlinebreakskip=0em plus 0.1em minus 0.01em
\XeTeXlinebreakpenalty=0
\renewcommand{\baselinestretch}{1.4}
\settowidth{\parindent}{あ}
%%%%
\usepackage{dcolumn,hhline,colortbl}
\usepackage[table]{xcolor}
%%%%% 色付表
\newcolumntype{G}{%
  >{\columncolor[gray]{0.9}}c}
\newcolumntype{O}{%
  >{\columncolor{orange}}c}
\newcolumntype{M}{%
  >{\columncolor{green}}c}
\newcolumntype{Y}{%
  >{\columncolor{yellow}}c}
\newcolumntype{C}{%
  >{\columncolor{cyan}}c}
%%%% hyperref
%\usepackage[dvipdfm,colorlinks=true,linkcolor=blue]{hyperref} 
\usepackage[colorlinks=true,linkcolor=blue]{hyperref} 
%%%% numbering
%%% ToC down to susubsections
\settocdepth{subsubsection}
%%% Numbering down to sections
\setsecnumdepth{subsection}
%%% 名前定義を適当に
\def\figurename{{図}}
\def\tablename{表}
\def\contentsname{目次}
\def\listfigurename{図目次}
\def\listtablename{表目次}
\def\refname{参考文献}
\def\bibname{参考文献}
\def\indexname{索引}
\def\appendixname{付録}
%%%%%%%%%% Verbatim
\DefineVerbatimEnvironment%
{simplev}{Verbatim}
{fontsize=\small}
\DefineVerbatimEnvironment%
{examplev}{Verbatim}
{frame=leftline,fontsize=\small}
%%%%%%%%%%
\definecolor{shadecolor}{gray}{0.9}
\newenvironment{vvtm}%
{\parskip=0pt\lineskip=0pt\begin{center}\begin{minipage}{0.8\textwidth}\begin{snugshade}}%
  {\end{snugshade}\end{minipage}\end{center}}
%\includeonly{appendix}
%%%%%%%%%%
\begin{document}
\headstyles{default}
\tightlists
\midsloppy
\raggedbottom
\chapterstyle{ell}
%%%%%%%%
\frontmatter
\pagestyle{empty}
% 表紙 %%%%%%%%%%%%%%%%%%%%%%%%%%%%%%%%%%%%%%%%%%%%%%%%%%%%%
\title{\textbf{PigNose 使用手引}\\ver 1.0 \\ --\textit{Draft}--}
\author{澤田 寿実\\
  info@cafeobj.org}
\date{}
\maketitle
  \begin{center}
    \includegraphics[scale=0.5]{pignose.pdf}
  \end{center}
\thispagestyle{empty}
\newpage
% 構成 %%%%%%%%%%%%%%%%%%%%%%%%%%%%%%%%%%%%%%%%%%%%%%%%%%%%%
\section*{はじめに}
本書は, CafeOBJ 言語で書かれた仕様のための自動定理証明システム
PigNose の利用手引である. 
システムは SRA版 CafeOBJ インタプリタ\footnote{今の所, CafeOBJ 言語のイ
  ンタプリタはこれしか存在しない. システムの入手先については第
  ~\ref{sec:distribution}節を参照されたい.} 
を拡張したものになっている. 
従って, これを用いるには CafeOBJ インタプリタの使用法
についても知っておく必要がある. 本書では, PigNose 特有のコマンドについて
のみ説明するので, 必要に応じて CafeOBJ インタプリタのマニュアル\footnote{CafeOBJ インタプリタのマニュアルは
   \url{http://cafeobj.org/download/} からダウンロード
   出来る}. 
を参照されたい.
また, ここでは CafeOBJ 言語についての説明も行わない. 本書では, CafeOBJ
言語については既知のものと仮定している. CafeOBJ 言語についての解説は, 文
献\cite{CafeRep}を参照して頂きたい. 
% PigNose は resolution 原理を用いた反駁法によって定理証明を行うシステムで
% あるが, これについての解説も行わない. 
% 必要に応じて, 参考文献 \cite{chang-lee} 等を参照して頂きたい.  

% \section*{本書の構成}

本文は4部構成でPigNoseシステムの使用法が説明されている.
第I部でシステムのインストール法を述べ, 第II部で反駁エンジン,
第III部で詳細化検証, 第IV部で安全性モデル検査について説明する.

\section*{バグレポート/提案}

システムの不具合に関する報告や, 改善提案等に関する
連絡は, 電子メイルで \texttt{info@cafeobj.org} まで頂きたい.

\newpage
% 目次 %%%%%%%%%%%%%%%%%%%%%%%%%%%%%%%%%%%%%%%%%%%%%%%%%%%%%
% \doparttoc
\tableofcontents
\newpage
%\listoffigures
%\newpage
\pagenumbering{arabic}
\mainmatter
% 本文 %%%%%%%%%%%%%%%%%%%%%%%%%%%%%%%%%%%%%%%%%%%%%%%%%%%%%
\include{part1}
\include{part2}
\include{part3}
\include{part4}
\include{part5}
\include{appendix}

\newpage

% BIB %%%%%%%%%%%%%%%%%%%%%%%%%%%%%%%%%%%%%%%%%%%%%%%%%%%%%%
\begin{center}
\begin{thebibliography}{99}\itemsep=0pt

\bibitem{chang-lee} Chang, C. and Lee. R.C.,
  \textsl{Symbolic Logic and Mechanical Theorem Proving},
  Academic Press, 1973

\bibitem{ha} Joseph Goguen and Grant Malcolm, ``A Hidden Agenda'', in
  \emph{Theoretical Computer Science}, Vol.245 No.1, 2000, pp.55--101

\bibitem{otter} William McCune, ``\textsc{Otter3.0} Reference Manual
  and Guide'', Technical Report ANL-94/6, Argonne National Laboratory,
  1994, available at \texttt{http://www-unix.mcs.anl.gov/AR/otter/}

\bibitem{cafeobj} A.T.Nakagawa, T. Sawada, and K. Futatsugi,
  ``\textsl{CafeOBJ Manual}'', avaiable at 
  \texttt{ftp://ftp.sra.co.jp/lang/CafeOBJ/Manual/}

\bibitem{CafeRep} R\u{a}zvan Diaconescu and Kokichi Futatsugi,
  \textsf{CafeOBJ} \emph{Report}. World Scientific, 1998

\end{thebibliography}
\end{center}
\end{document}

